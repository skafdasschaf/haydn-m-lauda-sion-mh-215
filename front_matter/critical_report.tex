\documentclass[parskip=full]{scrreprt}

\RedeclareSectionCommand[pagestyle=plain,afterskip=10pt plus 1pt]{chapter}
\setkomafont{chapter}{\mdseries\headingfont\fontsize{40}{40}\selectfont\color{black!80}}
\setkomafont{pageheadfoot}{\normalsize}

\def\pnumbox#1{#1\hspace*{8cm}}
\DeclareTOCStyleEntry[
  indent=0pt,
  beforeskip=0pt,
  entryformat=\itshape,
  entrynumberformat=\textcolor{oldred},
  numwidth=2em,
  linefill=\hfill,
  pagenumberbox=\pnumbox,
  pagenumberformat=\itshape
]{tocline}{part}

% use commented values if there are no parts
\DeclareTOCStyleEntry[
  indent=2em,%0pt,
  beforeskip=-\baselineskip,%0pt,
  entrynumberformat=\textcolor{oldred},
  numwidth=2em,
  linefill=\hfill,
  pagenumberbox=\pnumbox
]{tocline}{section}

\usepackage{polyglossia}
\setdefaultlanguage{english}

\usepackage{fontspec}

\setmainfont{Source Sans Pro}[
  ItalicFont = Source Sans Pro Italic,
  BoldFont = Source Sans Pro Bold,
  BoldItalicFont = Source Sans Pro Bold Italic,
  FontFace = {lt}{n}{Source Sans Pro Light},
  FontFace = {lt}{it}{Source Sans Pro Light Italic},
  FontFace = {sb}{n}{Source Sans Pro Semibold},
  FontFace = {sb}{it}{Source Sans Pro Semibold Italic},
  Numbers = Proportional,
  Ligatures = Common
]
\DeclareRobustCommand{\ltseries}{\fontseries{lt}\selectfont}
\DeclareRobustCommand{\sbseries}{\fontseries{sb}\selectfont}
\DeclareTextFontCommand{\textlt}{\ltseries}
\DeclareTextFontCommand{\textsb}{\sbseries}

\newfontfamily\headingfont{Fredericka the Great}

\usepackage[pass]{geometry}
\newgeometry{twoside,inner=20mm,outer=40mm,top=20mm,bottom=40mm}

\usepackage{url}
\urlstyle{same}

\usepackage{microtype}
\microtypesetup{verbose=silent}

\usepackage{booktabs,array,longtable}
\newcolumntype{L}[1]{>{\raggedright\let\newline\\\arraybackslash\hspace{0pt}}p{#1}}

\usepackage{graphicx}

\usepackage{xcolor}
\definecolor{oldred}{rgb}{.8313,0,0}

\usepackage{pdfpages}

\usepackage{scrlayer-scrpage}
\pagestyle{scrheadings}
\clearscrheadfoot
\cfoot[\thepage]{\thepage}
\pagenumbering{roman}

\usepackage{enumitem}
\setlist[description]{%
  style=nextline,%
  leftmargin=2em,%
  first=\ltseries,%
  font=\normalfont%
}
\def\lyrefitem#1{\item[\lyref{#1}]}


\makeatletter

\def\@firstofthree#1#2#3{#1}
\def\@secondofthree#1#2#3{#2}
\def\@thirdofthree#1#2#3{#3}
\def\Dotfill{\leavevmode\cleaders\hb@xt@ .75em{\hss .\hss }\hfill \kern \z@}

\def\lyrefnumber#1{\expandafter\@setref\csname r@#1\endcsname\@firstofthree{#1}}
\def\lyreftitle#1{\expandafter\@setref\csname r@#1\endcsname\@secondofthree{#1}}
\def\lyrefpage#1{\expandafter\@setref\csname r@#1\endcsname\@thirdofthree{#1}}

\def\lyref#1{%
  \begingroup%
  \makebox[0pt][l]{\color{oldred}\lyrefnumber{#1}}\hspace*{2em}%
  \lyreftitle{#1}\Dotfill\lyrefpage{#1}%
  \endgroup%
}
\InputIfFileExists{../out/lilypond.ref}{}{\InputIfFileExists{../lilypond.ref}{}{}}


\newcommand\fancytitlehead{
	\headingfont%
	\fontsize{80}{80}\selectfont\textcolor{black!80}{\@ifundefined{@shortname}{\@lastname}{\@shortname}.}\\[15pt]%
	\fontsize{60}{60}\selectfont\@ifundefined{@shorttitle}{\@title}{\@shorttitle}.%
}


\def\firstname#1{\def\@firstname{#1}}
\def\lastname#1{\def\@lastname{#1}}
\def\shortname#1{\def\@shortname{#1}}
\def\shorttitle#1{\def\@shorttitle{#1}}
\def\namesuffix#1{\def\@namesuffix{#1}}
\def\instrumentation#1{\def\@instrumentation{#1}}
\def\parts#1{\def\@parts{#1}}

\firstname{\relax}
\lastname{\relax}
\namesuffix{\relax}
\instrumentation{\relax}
\parts{\relax}


\def\maketitle{%
\begin{titlepage}%
	\Large%
	{\@titlehead}%
	\vfill%
	{\strut\@firstname}\\%
	{\sbseries\color{oldred}\strut\@lastname}\\%
	{\strut\@namesuffix}%
	\vfill%
	{\sbseries\@title}\\%
	{\@subtitle}\\[\baselineskip]%
	{\itshape\@instrumentation}%
	\vfill%
	{\itshape\@parts}\hspace*{\fill}\raisebox{0pt}[0pt][0pt]{\includegraphics{ees_logo}}%
\end{titlepage}%
}


\newif\iftemplate\templatetrue
\newif\ifprintreport\printreportfalse
\directlua{
scores = {
  ob1 = "Oboe I",
  ob2 = "Oboe II",
  ottoni = "Clarino I, II in C\string\\newline Timpani in C–G",
  vl1 = "Violino I",
  vl2 = "Violino II",
  vla = "Viola",
  coro = "Coro",
  org = "Organo",
  b = "Bassi",
  full_score = "Full Score"
}

last_arg = arg[\string#arg]
texio.write("Last argument: " .. last_arg)
if not (scores[last_arg] == nil) then
  tex.print("\string\\def\string\\lypdfname{" .. last_arg .. "}")
  tex.print("\string\\parts{" .. scores[last_arg] .. "}")
  if (last_arg == "full_score") then
    tex.print("\string\\printreporttrue")
  end
end
}

% uncomment if the full score is in landscape format
%\ifprintreport
%\geometry{landscape,outer=127mm}
%\fi

\@ifundefined{lypdfname}{%
  \templatefalse
  \printreporttrue
  \parts{Draft}
}{\templatetrue}

\makeatother






\begin{document}
\frenchspacing

\titlehead{\fancytitlehead}
\firstname{Johann Michael}
\lastname{Haydn}
\shortname{M. Haydn}
\title{Lauda Sion}
\subtitle{MH 215\\(A-Ed B 176)}
\instrumentation{S, A, T, B (solo), S, A, T, B (coro), 2 ob, 2 cor, 2 clno, timp, 2 vl, b, org}
\maketitle


\thispagestyle{empty}

\vspace*{\fill}

\raisebox{-4mm}{\includegraphics{byncsaeu}}\hspace*{1em}Wolfgang Esser-Skala, 2020

© 2020 by Wolfgang Esser-Skala. This edition is licensed under the Creative Commons Attribution-NonCommercial-ShareAlike 4.0 International License. To view a copy of this license, visit \url{http://creativecommons.org/licenses/by-nc-sa/4.0/}. 

Music engraving by LilyPond 2.18.0 (\url{http://www.lilypond.org}).\\
Front matter typeset with Source Sans Pro and Fredericka the Great.

\textit{First version, June 2020}

\vspace*{2cm}

\ifprintreport
\chapter*{Critical Report.}

This edition bases upon a copy in the Domarchiv Eisenstadt. The digital version of the manuscript is available at \url{http://dommusikarchiv.martinus.at/site/werkverzeichnis/gallery/976.html} (siglum B 176).

In general, this edition closely follows the manuscript. Any changes that were introduced by the editor are indicated by italic type (lyrics, dynamics and directions), parentheses (expressive marks and bass figures) or dashes (slurs and ties). Accidentals are used according to modern conventions. Asterisks denote changes that are clarified in the detailed remarks below.\footnote{Abbreviations: A, alto; B, bass; b, basses; clno, clarion; cor, horn; Ms, manuscript; ob, oboe; org, organ; r, rest; S,~soprano; T, tenor; timp, timpani; vl, violin.}

\bigskip

\begin{longtable}{ll L{10cm}}
	\toprule
	\itshape Bar & \itshape Staff & \itshape Note \\
	\midrule \endhead
	3   & vl 1  & grace note missing in Ms \\
	5   & vl 1  & grace note missing in Ms \\
	5   & T     & 3rd quarter in Ms: fis4 \\
	10  & S     & 8th sixteenth in Ms: a′16 \\
	12  & S     & grace note missing in Ms \\
	30  & S     & 8th sixteenth in Ms: fis′16 \\
	41  & T     & 6th eighth in Ms: g8 \\
	42  & cor 2 & grace note missing in Ms \\
	45  & S     & grace note missing in Ms \\
	51  & S     & 1st half of bar in Ms: b′4–cis″4 \\
	60  & S     & 8th sixteenth in Ms: e′16 \\
	62  & B     & last eighth in Ms: B8 \\
	64  & T     & last eighth in Ms: c′8 \\
	70  & S     & 8th sixteenth in Ms: fis′16 \\
	72  & S     & grace note missing in Ms \\
	85  & B     & 3rd quarter in Ms: e4 \\
	88  & T     & grace note missing in Ms \\
	90  & vl 2  & 3rd quarter in Ms: g′8–r16–g′16 \\
	92  & S     & 8th sixteenth in Ms: a′16 \\
	\midrule
	96  & T     & 1st quarter in Ms: b4 \\
	98  & T     & 1st note in Ms: b4. \\
	105 & T     & 5th eighth in Ms: c′8 \\
	108 & S     & grace note missing in Ms \\
	108 & cor 2 & 1st half of bar in Ms: g′4–r8 \\
	109 & T     & 2nd eighth in Ms: d′8 \\
	109 & B     & 2nd eighth in Ms: g8 \\
	110 & vl 1  & 8th sixteenth in Ms: fis″16 \\
	111 & A     & 2nd quarter in Ms: e′4 \\
	114 & cor 2 & bar in Ms: c′2. \\
	114 & A     & 1st quarter in Ms: f′4 \\
	128 & S     & grace note missing in Ms \\
	131 & B     & 5th eighth in Ms: g8 \\
	140 & S     & 3rd quarter in Ms: e″4 \\
	\midrule
	157 & T     & 3rd quarter in Ms: fis8–fis8 \\
	161 & A     & 7th eighth in Ms: g′8 \\
	162 & S     & grace note missing in Ms \\
	163 & T     & 4th eighth in Ms: b8 \\
	168 & A     & grace note missing in Ms \\
	172 & S     & 6th eighth in Ms: d″8 \\
	175 & T     & 3rd quarter in Ms: e′8 \\
	178 & org   & 5th eighth in Ms: g8 \\
	179 & T     & 2nd/3rd quarter in Ms: d′8–b8–g4 \\
	180 & T     & 4th quarter in Ms: g4 \\
	\bottomrule
\end{longtable}


This edition has been compiled and checked with utmost diligence. Nevertheless, errors and mistakes cannot be totally excluded. Please report any errors and mistakes to \url{wolfgang@esser-skala.at} or create an issue or pull request on the edition’s GitHub page \url{https://github.com/skafdasschaf/haydn-m-lauda-sion-mh-215}. Your help will be greatly appreciated.

\bigskip
\textit{Salzburg, June 2020\\
Wolfgang Esser-Skala}

\cleardoublepage
\chapter*{Contents.}

\InputIfFileExists{../out/lilypond.toc}{}{\InputIfFileExists{../lilypond.toc}{}{}}

Lauda Sion Salvatorem,
lauda ducem et pastorem
in hymnis et canticis.

Quantum potes, tantum aude,
quia maior omni laude,
nec laudare sufficis.

Laudis thema specialis
panis vivus et vitalis
hodie proponitur.

Quem in sacrae mensa coenae
turbae fratrum duodenae
Ddatum non ambigitur.

Sit laus plena, sit sonora,
sit iucunda, sit decora
mentis iubilatio,

dies enim solemnis agitur
in qua mensae prima recolitur
huius institutio.

In hac mensa novi Regis
novum Pascha novae legis
phase vetus terminat.

Vetustatem novitas,
umbram fugat veritas,
noctem lux eliminat.

Quod in coena Christus gessit,
faciendum hoc expressit
in sui memoriam:

Docti sacris institutis
panem, vinum in salutis
consecramus hostiam.

Dogma datur Christianis,
quod in carnem transit panis
et vinum in sanguinem.

Quod non capis, quod non vides,
animosa firmat fides
praeter rerum ordinem.

Sub diversis speciebus,
signis tantum et non rebus,
latent res eximiae:

Caro cibus, sanguis potus,
manet tamen Christus totus
sub utraque specie.

A sumente non concisus,
non confractus, non divisus
integer accipitur.

Sumit unus, sumunt mille,
quantum isti, tantum ille,
nec sumptus consumitur.

Sumunt boni, sumunt mali,
sorte tamen inaequali,
vitae vel interitus.

Mors est malis, vita bonis,
vide paris sumptionis
quam sit dispar exitus

fracto demum sacramento,
ne vacilles, sed memento
tantum esse sub fragmento,
quantum toto tegitur.

Nulla rei fit scissura,
signi tantum fit fractura,
qua nec status nec statura
signati minuitur.

Ecce panis Angelorum,
factus cibus viatorum,
vere panis filiorum,
non mittendus canibus!

In figuris praesignatur,
cum Isaac immolatur,
agnus Paschae deputatur,
datur manna patribus.

Bone pastor, panis vere,
Jesu, nostri miserere,
tu nos pasce, nos tuere,
tu nos bona fac videre
in terra viventium.

Tu qui cuncta scis et vales,
qui nos pascis hic mortales,
tuos ibi commensales,
cohaeredes et sodales
fac sanctorum civium.
Amen alleluja.

\cleardoublepage
\fi

\iftemplate
\includepdf[pages=-]{../out/\lypdfname.pdf}
\fi

\end{document}